\documentclass[letter,twoside,11pt]{article}

\usepackage[spanish]{babel}
\usepackage[utf8]{inputenc}

\usepackage{lmodern}
\usepackage[T1]{fontenc}
\usepackage{textcomp}

\usepackage{graphicx}
\usepackage{pstricks}

\usepackage{anysize}
\marginsize{3cm}{2cm}{2cm}{3cm}

\usepackage{fancyhdr}
\usepackage{lastpage}
\pagestyle{fancy}

\fancyhf{}
\fancyhead[LE,RO]{SCESI \\
Sociedad Científica de Estudiantes de Sistemas e Informática}
\fancyfoot[CO,CE]{\thepage\ de \pageref{LastPage}}

\special{papersize=215.9mm,279.4mm}

\newcommand{\blankpage}{
\newpage
\thispagestyle{empty}
\mbox{}
\newpage
}

\renewcommand{\arraystretch}{1.2}

\title{\bf Pleni}
\author{
    Carlos Eduardo Caballero Burgoa \\
    Gonzalo Nina Mamani \\
    Cristhian Lima Saravia \\
}

\begin{document}
\maketitle

\blankpage
\tableofcontents
\blankpage

\section{Introducción}
Cualquier organización que expone sus servicios informáticos a redes de acceso
tendrán que realizar un esfuerzo significativo para asegurar que la información
y recursos estén protegidos. Internet es un factor primordial en la
comunicación, pero también un evidente riesgo potencial de acceso y mal uso de
los servicios e información disponibles.

Obviamente, se catalogan sistemas mas críticos que otros donde su seguridad debe
de ser muy significativa, pero en general todas las aplicaciones web deben de
estar protegidas y aseguradas ante los principales ataques.

En una aplicación web, dividimos la seguridad en:

\begin{description}
    \item [Disponibilidad] Propiedad o característica de los activos consistente
        en que las entidades o procesos autorizados tienen acceso a los mismos
        cuando lo requieren.
    \item [Integridad] Propiedad o característica consistente en que el activo
        de información no ha sido alterado de manera no autorizada.
    \item [Confidencialidad] Propiedad o característica consistente en que la
        información ni se pone a disposición, ni se revela a individuos,
        entidades o procesos no autorizados.
\end{description}

\section{Antecedentes}
Un efecto secundario del crecimiento exponencial que ha tenido el Internet, es
la privacidad de información tanto personal como profesional. En Internet
encontramos funcionando tiendas en línea, negocios que mueven grandes
cantidades de dinero, redes de los servicios que habilitan el comercio a nivel
internacional así como sitios de redes sociales que contienen información muy
delicada de la vida privada de sus miembros.

Mientras más se conecta el mundo, la necesidad de seguridad en los
procedimientos usados para compartir la información se vuelve más importante.
Desde muchos puntos de vista, podemos creer sin dudar que el punto más crítico
de la seguridad del Internet, lo tienen las piezas que intervienen de forma
directa con las masas de usuarios, es decir, las aplicaciones web.

Ahora que sabemos que la mayoría de los problemas de seguridad en los sitios web
se encuentran a nivel de aplicación y que son el resultado de escritura
defectuosa de código, debemos entender que programar aplicaciones web seguras
no es una tarea fácil, ya que requiere por parte del programador, no únicamente
mostrar atención en cumplir con el objetivo funcional básico de la aplicación,
sino una concepción general de los riesgos que puede correr la información
contenida, solicitada y recibida por el sistema.

\section{Justificación}
De manera social la SCESI brinda un servicio a la comunidad universitaria, y
entidades externas a la universidad, ayudando a mejorar la seguridad de la
sociedad y divulgando sobre lo que esto conlleva.

\section{Planteamiento del problema}
Existen muchos problemas que pueden ser identificados rápidamente desde el punto
de vista de un usuario normal (cliente) en el sitio web, y así brindar una
solución rápida y efectiva para así evitar dañar la imagen de su organización o
afectar adversamente a sus clientes, socios y empleados.

En el área de la seguridad se podrían detallar varios tipos de ataque como:

\begin{description}
    \item [Defacement de sitios web] Usuarios maliciosos deambulan en Internet
        específicamente para desfigurar sitios. A menudo esos sitios contienen
        lenguaje o imágenes ofensivas y el probable resultado es una imagen
        dañada.
    \item [Explotaciones web] A menudo, los atacantes comprometen un sitio web e
        instalan explotaciones para atacar a los visitantes del sitio. Estos son
        clasificados a menudo como defacement silenciosos ya que el sitio no se
        ve cambiado visualmente. Sophos señaló que la vasta mayoría de los
        sitios web que alojan malware (cerca de 80\%) son sitios legítimos que
        han sido comprometidos.
    \item [Fugas de información sensible] Los sitios web pueden filtrar
        información sensible mediante mensajes de error detallados, archivos que
        no tienen la intención de ser mostrados públicamente.
    \item [Fallos del sistema] También existen una serie de problemas:
        \begin{itemize}
        \item Enlaces rotos.
        \item Mensajes de error y advertencia.
        \item Servidores mal configurados o con la configuración por defecto.
        \item Certificados SSL expirados.
        \item Errores de servidor.
        \end{itemize}
\end{description}

Por lo mencionado se define el problema como:

\emph{“El elevado numero de sitios web que no poseen las características mínimas
de seguridad conduce a la obtención de información privilegiada.”}

\section{Objetivos}

\subsection{Objetivo general}

Desarrollar una herramienta de seguridad para la detección temprana de
actividades sospechosas de modo que pueda garantizarse un mínimo de seguridad
aceptable.

\subsection{Objetivos específicos}
\begin{itemize}
\item Registrar toda la información disponible del sitio web para el análisis
    posterior.
\item Crear un conjunto básico de rutinas que permitan verificar de manera
    periódica la integridad del sitio web.
\item Analizar la información registrada para encontrar algún tipo de código
    malicioso.
\item Diseñar las rutinas que permitan realizar avisos y/o alertas de los
    eventos para el correcto mantenimiento del sitio.
\end{itemize}

\section{Hipótesis o idea a defender}
Se pretende concientizar a la sociedad acerca de los peligros de la mala
administración de los sistemas y su información, para así no tener perdidas
importantes para cualquier organización.
De esta manera se quiere brindar soluciones a los problemas de seguridad mas
comunes.

\section{Aporte científico}
A partir de la información registrada podrían proyectarse errores frecuentes
referentes al proceso de desarrollo, y administración de forma que estos puedan
ser previstos en etapas mas tempranas.

\section{Diseño metodológico y teórico}
Para el proceso de creación de esta herramienta, se han seguido algunos
lineamientos fundamentales del desarrollo ágil, a pesar de eso, se fundamentan
muchos de los procedimientos creados basados en la experiencia en seguridad
informática de los componentes del equipo.

\section{Desarrollo del proyecto}
El desarrollo del proyecto se realizó en las siguientes etapas:

\subsection{Captación de la información}
Se procedieron a crear los métodos necesarios para absorber la mayor cantidad de
información disponible de forma publica del sitio a analizar.

También se crearon algunos otros métodos para el aprovechamiento del buscador
google en lo referente a la obtención de información.

\subsection{Visualización de la información captada}
Se aprovecharon algunas librerías Javascript para crear una visualización
apropiada de la información captada.

\subsection{Análisis de la información}
Con toda la información captada en las anteriores etapas, se crearon
herramientas para probar las distintas técnicas de vulneración del sitio web, de
forma que estas pudiesen ser mas automatizadas.

\subsection{Herramientas utilizadas}
Las herramientas necesarias para realizar este proyecto fueron básicamente las
siguientes:

\begin{itemize}
    \item Base de Datos NoSQL: CouchDB.
    \item Node.js.
    \item Express.js (framework para el desarrollo web).
    \item Servidor web para el despliegue: nginx.
    \item Base de datos para almacenamiento de sesiones: Redis.
    \item Librería D3 para visualización de grandes volúmenes de datos.
\end{itemize}

\section{Conclusiones y recomendaciones}
Pudo observarse con la herramienta desarrollada, que la gran mayoría de sitios
desarrollados a nivel local carecen de políticas apropiadas de seguridad de su
información, todo esto a causa de la dejadez de los encargados de
administración. Los cuales velan mas por el despliegue rápido que por el
mantenimiento continuo del sistema.

Creemos que la solución debe pasar por la especialización de las funcionalidades
de un sistema, en lugar de desear una solución que tenga muchas funcionalidades
y no cumpla con las especificaciones deseadas.

\begin{thebibliography}{99}

    \bibitem{NSIA} ThreatFactor NSIA.\\
    Extraído el 17 de Julio del 2014, de\\
    http://threatfactor.com/Products/NSIA/Features.

    \bibitem{UNAM} UNAM-CERT: Equipo de Respuesta a Incidentes UNAM.\\
    Aspectos básicos de la seguridad en aplicaciones web.\\
    Extraído el 17 de Julio del 2014, de\\
    http://www.seguridad.unam.mx/documento/?id=17\\

    \bibitem{Andalucia} Marco de desarrollo de la Junta de Andalucia.\\
    Conceptos de seguridad en aplicaciones WEB.\\
    http://www.juntadeandalucia.es/servicios/madeja/contenido/recurso/212\\
\end{thebibliography}

\end{document}

