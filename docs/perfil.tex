\documentclass[letter,twoside,11pt]{article}

\usepackage[spanish]{babel}
\usepackage[utf8]{inputenc}

\usepackage{lmodern}
\usepackage[T1]{fontenc}
\usepackage{textcomp}

\usepackage{graphicx}
\usepackage{pstricks}

\usepackage{anysize}
\marginsize{3cm}{2cm}{2cm}{3cm}

\usepackage{fancyhdr}
\usepackage{lastpage}
\pagestyle{fancy}

\fancyhf{}
\fancyhead[LE,RO]{SCESI \\
Sociedad Científica de Estudiantes de Sistemas e Informática}
\fancyfoot[CO,CE]{\thepage\ de \pageref{LastPage}}

\special{papersize=215.9mm,279.4mm}

\newcommand{\blankpage}{
\newpage
\thispagestyle{empty}
\mbox{}
\newpage
}

\renewcommand{\arraystretch}{1.2}

\title{\bf Pleni}
\author{
    Carlos Eduardo Caballero Burgoa \\
    Gonzalo Nina Mamani \\
    Christian Lima Saravia \\
}

\begin{document}
\maketitle

\blankpage
\tableofcontents
\blankpage

\section{Introducción}
Cualquier organización que expone sus servicios informáticos a redes de acceso
tendrán que realizar un esfuerzo significativo para asegurar que la información
y recursos esten protegidos. Internet es un factor primordial en la
comunicación, pero también un evidente riesgo potencial de acceso y mal uso de
los servicios e información disponibles.

Obviamente, se catalogan sistemas mas críticos que otros donde su seguridad debe
de ser muy significativa, pero en general todas las aplicaciones web deben de
estar protegidas y aseguradas ante los principales ataques.

En una aplicación web, dividimos la seguridad en:

\begin{description}
    \item [Disponibilidad] Propiedad o característica de los activos consistente
        en que las entidades o procesos autorizados tienen acceso a los mismos
        cuando lo requieren.
    \item [Integridad] Propiedad o característica consistente en que el activo
        de información no ha sido alterado de manera no autorizada.
    \item [Confidencialidad] Propiedad o característica consistente en que la
        información ni se pone a disposición, ni se revela a individuos,
        entidades o procesos no autorizados.
\end{description}

\section{Antecedentes}
Un efecto secundario del crecimiento exponencial que ha tenido el Internet, es
la privacidad de información tanto personal como profesional. En Internet
encontramos funcionando tiendas en línea, negocios que mueven grandes
cantidades de dinero, redes de los servicios que habilitan el comercio a nivel
internacional así como sitios de redes sociales que contienen información muy
delicada de la vida privada de sus miembros.

Mientras más se conecta el mundo, la necesidad de seguridad en los
procedimientos usados para compartir la información se vuelve más importante.
Desde muchos puntos de vista, podemos creer sin dudar que el punto más crítico
de la seguridad del Internet, lo tienen las piezas que intervienen de forma
directa con las masas de usuarios, es decir, las aplicaciones web.

Ahora que sabemos que la mayoría de los problemas de seguridad en los sitios web
se encuentran a nivel de aplicación y que son el resultado de escritura
defectuosa de código, debemos entender que programar aplicaciones web seguras
no es una tarea fácil, ya que requiere por parte del programador, no únicamente
mostrar atención en cumplir con el objetivo funcional básico de la aplicación,
sino una concepción general de los riesgos que puede correr la información
contenida, solicitada y recibida por el sistema.

\section{Definición del problema}
Existen muchos problemas que pueden ser identificados rapidamente desde el punto
de vista de un usuario normal (cliente) en el sitio web, y asi brindar una
solución rapida y efectiva para asi evitar dañar la imagen de su organización o
afectar adversamente a sus clientes, socios y empleados tales como:

\begin{description}
    \item [Defacement de sitios web] Usuarios maliciosos deambulan en Internet
        específicamente para desfigurar sitios. A menudo esos sitios contienen
        lenguaje o imágenes ofensivas y el probable resultado es una imagen
        dañada.
    \item [Explotaciones web] A menudo, los atacantes comprometen un sitio web e
        instalan explotaciones para atacar a los visitantes del sitio. Estos son
        clasificados a menudo como defacement silenciosos ya que el sitio no se
        ve cambiado visualmente. Sophos señaló que la vasta mayoría de los
        sitios web que alojan malware (cerca de 80\%) son sitios legítimos que
        han sido comprometidos.
    \item [Fugas de información sensible] Los sitios web pueden filtrar
        información sensible mediante mensajes de error detallados, archivos que
        no tienen la intención de ser mostrados públicamente.
    \item [Fallos del sistema] También existen una serie de problemas:
        \begin{itemize}
        \item Enlaces rotos.
        \item Mensajes de error y advertencia.
        \item Servidores mal configurados o con la configuración por defecto.
        \item Certificados SSL expirados.
        \item Errores de servidor.
        \end{itemize}
\end{description}

\section{Objetivo general}
Identificar problemas de seguridad de sitios web de manera similar a un motor de
búsqueda. Analizar el contenido descubierto mediante el uso de un motor de
análisis que es capaz de detectar amenazas tanto conocidas como desconocidas,
cambios inusuales y contenido anómalo.

Una vez que se detecta un problema, se invoca el sistema de respuesta a
incidentes. La respuesta a incidentes realiza acciones de manera rapida y
temprana informando mediante el envío de mensajes (mensajería instantánea,
correo electrónico, SMS) al personal encargado de la aplicacion web.

\section{Objetivos específicos}
\begin{itemize}
\item Crear una aplicacion web que permita realizar una copia del sitio web a
    analizar y almacenarlo en una base de datos.
\item Crear un conjunto básico de rutinas que permitan analizar el contenido en
    busca de codificación sospechosa.
\item Crear un conjunto básico de rutinas que permitan verificar de manera
    periodica la integridad del sitio web, mediante la comparacion de algoritmos
    de hash (md5 y sha).
\item Diseñar los modulos que permitan realizar alertas y/o avisos de los
    eventos.
\end{itemize}

\section{Herramientas}
Las herramientas necesarias para realizar este proyecto es basicamente las
siguientes:

\begin{itemize}
\item Hosting: Se necesita un espacio donde almacenar el código que soporte las
    siguientes tecnologias:
    \begin{itemize}
    \item Base de Datos NoSQL: MongoDB.
    \item Node.js
    \item Servidor Web: nginx.
    \end{itemize}
\end{itemize}

\section{Justificación}
Este proyecto tiene connotaciones de manera academica brindando el aprendizaje
de programacion de aplicaciones web a todos los participantes del grupo asi
tambien asimilar las medidas para detectar fallos y politicas de seguridad
respecto a aplicaciones web.

De manera social la SCESI brindará un servicio a la comunidad universitaria, y
entidades externas a la universidad, ayudando a mejorar la seguridad de la
sociedad y concientizando sobre lo que esto conlleva.

\begin{thebibliography}{99}

    \bibitem{NSIA} ThreatFactor NSIA.\\
    Extraído el 17 de Julio del 2014, de\\
    http://threatfactor.com/Products/NSIA/Features.

    \bibitem{UNAM} UNAM-CERT: Equipo de Respuesta a Incidentes UNAM.\\
    Aspectos básicos de la seguridad en aplicaciones web.\\
    Extraído el 17 de Julio del 2014, de\\
    http://www.seguridad.unam.mx/documento/?id=17\\

    \bibitem{Andalucia} Marco de desarrollo de la Junta de Andalucia.\\
    Conceptos de seguridad en aplicaciones WEB.\\
    http://www.juntadeandalucia.es/servicios/madeja/contenido/recurso/212\\
\end{thebibliography}

\end{document}

